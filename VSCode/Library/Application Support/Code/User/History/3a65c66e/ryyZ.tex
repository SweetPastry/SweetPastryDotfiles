
\documentclass[a4paper,11pt]{amsart}

\usepackage{gensymb}

\usepackage{tikz}
\usetikzlibrary{arrows.meta}
\usepackage{caption}

\usepackage{times}
\usepackage[top=27mm, left=23mm, bottom=23mm, right=23mm]{geometry}
\usepackage{amsfonts, amssymb, amsgen, amsthm, amscd, amsmath}
\usepackage{mathtools}
\mathtoolsset{showonlyrefs=true}

\usepackage{bm}
%\usepackage{mathpazo}
\usepackage{domitian}
\usepackage[T1]{fontenc}
\let\oldstylenums\oldstyle

\usepackage{enumerate}
\usepackage{color}
\usepackage[all]{xy}

\newtheorem{theorem}{Theorem}[section]
\newtheorem{proposition}[theorem]{Proposition}
\newtheorem{lemma}[theorem]{Lemma}
\newtheorem{corollary}[theorem]{Corollary}
\newtheorem{claim}[theorem]{Claim}
\theoremstyle{definition}
\newtheorem{remark}[theorem]{Remark}
\newtheorem{example}[theorem]{Example}
\newtheorem{definition}[theorem]{Definition}

\newcommand{\outimes}[2]{\overset{#1}{\underset{#2}{\otimes}}}
\newcommand{\C}[1]{\mathcal{#1}}
\newcommand{\B}[1]{\mathbb{#1}}
\newcommand{\G}[1]{\mathfrak{#1}}
\newcommand{\rmod}[1]{\text{{\bf Mod}-}{#1}}

\newcommand{\Span}{\text{\rm Span}}
\newcommand{\Tor}{\text{\rm Tor}}
\newcommand{\Ind}{\text{\rm Ind}}
\newcommand{\Res}{\text{\rm Res}}
\newcommand{\Ext}{\text{\rm Ext}}
\newcommand{\Hom}{\text{\rm Hom}}
\newcommand{\CoInd}{\text{\rm CoInd}}
\newcommand{\Simp}{{\Delta}}
\newcommand{\Diff}{{\Omega}}
\newcommand{\xla}[1]{\xleftarrow{#1}}
\newcommand{\colim}{\text{colim}}
\renewcommand{\baselinestretch}{1.15}

\setlength{\parskip}{1.2mm}
\setlength{\parindent}{0mm}

\title{Thermodynamics Assignment For The Fifth Time}

\author{Haixuan Lin - 23307110267}
\email{23307110267@m.fudan.edu.cn}


\address{Fudan University, Physics Department, China}

\begin{document}
	
	\begin{abstract}
		Here is the thermodynamics assignment for the fifth time which is for the course given by professor Yuanbo Zhang. In order to practise the expertise in scientific film of physics, students need to practise using \LaTeX to composing their own work, even if this is only a ordinary homework.
	\end{abstract}
	
	\maketitle
	\section*{Main Text}
	
	\subsection*{4.4.2}
	
	\subsubsection*{(1)}
	
	This is a quasi-static process, using the van der Waals equation to approximate the relationship between volume and pressure
	
	$$
	A=\int_{V_1}^{V_2}{\left( \frac{RT}{V_m-b}-\frac{a}{V_{m}^{2}} \right) \mathrm{d}V_m=RT\ln \frac{V_{2,m}-b}{V_{1,m}-b}+\frac{a}{V_2,m}}-\frac{a}{V_{1,m}}
	$$
	
	\subsubsection*{(2)}
	
	Since it is an equal volume process, there is no work, so the heat absorption is equal to the change in internal energy.
	
	$$
	Q=\Delta U=c\Delta T
	$$
	
\end{document}
