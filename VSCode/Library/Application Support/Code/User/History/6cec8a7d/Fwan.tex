\documentclass[
    style = 0,
    lang = en,
    % bibstyle = apa
]{spBeamer}
\spTitle{\texttt{spBeamer} Document}
\spAuthor{Sweet Pastry}
\spAuthorInShort{SP}
\spAffiliation{Fudan University, Shanghai, China}
\spAffiliationInShort{FDU}

\begin{document}
    \section{How to use it}
        \subsection{Preamble and Info Command}
            \begin{frame}[fragile]{Preamble}
                In the preamble, please provide the following details to complete your Beamer presentation setup:
                \vspace{-5pt}
                \begin{Verbatim}[xleftmargin=-115pt]
                    \documentclass[
                        style = 2, % default o
                        bibstyle = apa, % if you need apa
                        lang = cn, % if you write in Chinese
                    ]{spBeamer}

                    \spAuthor{Your name}
                    \spAuthorInShort{Your name in short}
                    \spTitle{This Beamer's title}
                    \spSubtitle{This Beamer's subtitle if you need}
                    \spAffiliation{Your affiliation}
                    \spAffiliationInShort{Your affiliation in short if you need}
                    \spDate{default `\today`}
                \end{Verbatim}
            \end{frame}

            \begin{frame}{Some clarifications}
                \textbf{Q}: What is the difference between \texttt{\textbackslash spAuthor} and \texttt{\textbackslash spAuthorInShort}? Similarly, what distinguishes \texttt{\textbackslash spAffiliation} from \texttt{\textbackslash spAffiliationInShort}?

                \textbf{A}: "InShort" will be used in footline.
            \end{frame}

        \subsection{The options}
            \begin{frame}[fragile]{Options}
                The value in the right of = is default value.
                \begin{Verbatim}[xleftmargin=-100pt]
                    lang = en % english mode default
                    style = 0 % DarkRed style default
                    bibstyle = ieee & gb7714-2015 % when en and cn
                    ref = ref % if your .bib file has other name, change it
                    colorlinks = true
                    nocite = true
                \end{Verbatim}
            \end{frame}

    % \section{Some example}
    %     \begin{frame}
    %         Almost every feature in \texttt{spArticle} is also supported in \texttt{spBeamer}.
    %     \end{frame}

    %     \subsection{Math}
    %         \begin{frame}{math}
    %             \begin{equation}
    %                 \langle x_f, t_f \,\vert\, x_i, t_i \rangle
    %                 \;=\;
    %                 \int \mathcal{D}[x(t)] \;\exp\!\biggl(\tfrac{i}{\hbar} S[x(t)]\biggr),
    %             \end{equation}
    %             \begin{equation}
    %                 \gamma_{\mathrm{Berry}}
    %                 = i \int_{C}
    %                 \bigl\langle \psi(\lambda) \mid \boldsymbol{\nabla}_{\lambda}\,\psi(\lambda) \bigr\rangle
    %                 \cdot \mathrm{d}\lambda,
    %             \end{equation}
    %         \end{frame}

    %     \subsection{\texttt{tikz}}
    %         \subsubsection{normal}
    %             \begin{frame}{normal tikz}
    %                 \begin{center}
    %                     \begin{tikzpicture}[x=0.75pt,y=0.75pt,yscale=-1,xscale=1]
    %                         \draw [color={rgb, 255:red, 73; green, 135; blue, 206 }  ,draw opacity=1 ][line width=1.5]  (246,173) -- (485,173)(266,12) -- (266,195) (478,168) -- (485,173) -- (478,178) (261,19) -- (266,12) -- (271,19) (297,168) -- (297,178)(328,168) -- (328,178)(359,168) -- (359,178)(390,168) -- (390,178)(421,168) -- (421,178)(452,168) -- (452,178)(261,142) -- (271,142)(261,111) -- (271,111)(261,80) -- (271,80)(261,49) -- (271,49) ;
    %                         \draw   ;
    %                         \draw  [color={rgb, 255:red, 70; green, 155; blue, 36 }  ,draw opacity=1 ][line width=1.5]  (247,189) .. controls (298.67,-51) and (350.33,329) .. (402,89) ; 
    %                         \draw  [color={rgb, 255:red, 209; green, 53; blue, 53 }  ,draw opacity=1 ][line width=1.5]  (328,31) .. controls (369.67,217.67) and (411.33,217.67) .. (453,31) ;
    %                         \draw [color={rgb, 255:red, 141; green, 34; blue, 137 }  ,draw opacity=1 ][line width=1.5]    (221,30) -- (505,192) ;
    %                         \draw  [line width=0.75]  (391.4,127.4) .. controls (393.75,123.37) and (392.91,120.18) .. (388.88,117.83) -- (381.48,113.51) .. controls (375.72,110.15) and (374.02,106.45) .. (376.37,102.42) .. controls (374.02,106.45) and (369.96,106.79) .. (364.21,103.43)(366.8,104.94) -- (357.78,99.68) .. controls (353.75,97.33) and (350.55,98.17) .. (348.2,102.2) ;
    %                         \draw (286,49) node  [color={rgb, 255:red, 146; green, 29; blue, 130 }  ,opacity=1 ,rotate=-30.96]  {$A_{1}$};
    %                         \draw (469,151) node  [color={rgb, 255:red, 145; green, 25; blue, 123 }  ,opacity=1 ,rotate=-30.96]  {$A_{2}$};
    %                         \draw (234,193) node  [color={rgb, 255:red, 36; green, 114; blue, 18 }  ,opacity=1 ,rotate=-14.47]  {$y=x^{3}$};
    %                         \draw (365,42) node  [color={rgb, 255:red, 179; green, 35; blue, 24 }  ,opacity=1 ,rotate=-344.74]  {$y=x^{2}$};
    %                         \draw (382.8,93.2) node  [font=\footnotesize,rotate=-22.93]  {$d$};
    %                     \end{tikzpicture}
    %                 \end{center}
    %             \end{frame}

    %         \subsection{\texttt{tikz-cd}}
    %             \begin{frame}[fragile]{\texttt{tikz-cd}}
    %                 \begin{center}
    %                     \begin{tikzcd}[column sep=tiny]
    %                         & \pi_1(U_1) \ar[dr] \ar[drr, "j_1", bend left=20]
    %                         &
    %                         &[1.5em] \\
    %                         \pi_1(U_1\cap U_2) \ar[ur, "i_1"] \ar[dr, "i_2"']
    %                         &
    %                         & \pi_1(U_1) \ast_{ \pi_1(U_1\cap U_2)} \pi_1(U_2) \ar[r, dashed, "\simeq"]
    %                         & \pi_1(X) \\
    %                         & \pi_1(U_2) \ar[ur]\ar[urr, "j_2"', bend right=20]
    %                         &
    %                         &
    %                         \end{tikzcd}
    %                 \end{center}
    %             \end{frame}
            
    %         \subsection{\texttt{circuitikz}}
    %             \begin{frame}[fragile]{\texttt{circuitikz}}
    %                 \begin{center}
    %                     \ctikzset{amplifiers/fill=cyan!15, resistors/fill=magenta!15}
    %                     \begin{circuitikz}[european]
    %                         \draw (0, 0) node[above]{$v_i$} to[short, o-] ++(1, 0) node[op amp, noinv input up, anchor=+](OA){\texttt{OA}} (OA.-) -- ++(0, -1) coordinate(FB) to[R=$R1$] ++(0, -2) node[ground]{} (FB) to[R=$R_2$, *-] (FB -| OA.out) -- (OA.out) to[short, *-o] ++(1, 0) node[above]{$v_o$}; 
    %                     \end{circuitikz}
    %                 \end{center}
    %             \end{frame}

    %         \subsection{chem}
    %             \begin{frame}[fragile]{\texttt{mhchem} and \texttt{chemfig}}
    %                 \begin{gather}
    %                     \ce{Zn^2+
    %                     <=>[+ 2OH-][+ 2H+]
    %                     $\underset{\text{amphoteres Hydroxid}}{\ce{Zn(OH)2 v}}$
    %                     <=>[+ 2OH-][+ 2H+]
    %                     $\underset{\text{Hydroxozikat}}{\ce{[Zn(OH)4]^2-}}$
    %                     }\nonumber
    %                     \\\nonumber\\
    %                     \ce{x Na(NH4)HPO4 ->[\Delta] (NaPO_3)_x + x NH_3 ^ + x H_2O}\nonumber
    %                     \\\nonumber\\
    %                     \ce{Hg^2+ ->[I-] HgI2 ->[I-] Hg^{II}I4^2-}
    %                 \end{gather}
    %             \end{frame}

    %             \begin{frame}
    %                 \begin{center}
    %                     \begin{tikzpicture}
    %                         \node at (0.78,2.95) {\chemfig{
    %                                 <[:-25,1.176]
    %                                 (   
    %                                     -[0,0.01,,,line width=6.1pt,shorten <=-1.5pt,shorten >=-1.5pt]           
    %                                     >[:25,1.176] 
    %                                     -[:135,0.8] (
    %                                             -[:30,1]-[:-35,1]OH
    %                                             )
    %                                     -[180,1]
    %                                     -[:-135,0.8]
    %                                 )}};
    %                         \node at (0,0.3){\chemfig{
    %                                     -[:25,1.176]
    %                                 (            
    %                                     -[:-25,1.176]
    %                                     <[:-135,0.8]
    %                                     -[180,,,,line width=4.8pt,shorten <=-2.2pt,shorten >=-2.2pt]
    %                                     >[:135,0.8]            
    %                                 )}};
    %                         \draw [dashed]  (0,1.75) -- (0,2.9);
    %                         \draw [dashed]  (0,0.3) -- (0,1.3);
    %                         \draw (0,0.3) ellipse (0.75cm and 0.35cm);
    %                         \draw (0,2.85) ellipse (0.75cm and 0.35cm);
    %                         \node at (0,1.5) {\Large Fe};
    %                     \end{tikzpicture}
    %                 \end{center}
    %             \end{frame}

    %             \begin{frame}
    %                 \begin{center}
    %                     \chemfig{%
    %                     OH-[@{l, .75}]P(=[2, .6]O)(-[6, .6]OH)-[,0.6]O-[@{r, .25}:-30]-[:25,1.176]O
    %                     (            
    %                         -[:-25,1.176](-[2, .75]N([::-18]
    %                             *5([,.75]-(
    %                                 *6(-N=-N=(-[2, 0.5]NH_2)-=))--N=-)))
    %                         <[:-135,0.8](-[6, .75]OH)
    %                         -[180,,,,line width=4.8pt,shorten <=-2.2pt,shorten >=-2.2pt](-[6, .75]OH)
    %                         >[:135,0.8]    
    %                     )
    %                     }
    %                     \polymerdelim[height=25pt, depth=25pt, indice=\!\!3]{l}{r}
    %                 \end{center}
    %             \end{frame}
    % \section{Thanks to, I learn a lot from them!}
\end{document}