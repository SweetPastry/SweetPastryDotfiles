\spChapter{前言}
    本文件是笔者在本科二三年级时学习\emph{群论}一课时使用的笔记. 主要参考了《群论及其在固体物理中的应用》\cite{Xu1994}一书, 参考以本学校万义顿教授数理方法的讲义\cite{Wan2024}, 本讲义从对称性的角度借以群论为数学工具重新讨论传统数理方法, 北京大学李新征老师相关教学视频 (发布于\href{https://www.bilibili.com/video/BV1Ux4y177BH?vd_source=d630661fd733349495a9252445d0c4a4}{哔哩哔哩【群论 (物理学) -北京大学-李新征】}) 以及网络上分享的讲义.

    另外一些参考资料主要是略读, 包括诸多国内外经典著作, 列举在下:
    \begin{enumerate}
        \item \textit{LIE GROUPS IN PHYSICS} by \textit{M.J.G. Veltman}.
        \item \textit{GT Lecture Notes} by \textit{Gregory W. Moore}.
    \end{enumerate}

    

    \vspace{2cm}
    \begin{flushright}
        \textit{林海轩} \\[1em]
        \textit{复旦大学\ \today}
    \end{flushright}