\documentclass[
    lang=cn,
]{spArticle}
\spTitle{%
    符号资本与文化记忆\\
    \hspace{12em}\Large{——圣依纳爵主教座堂的多重拆解}
}
\spAuthor{林海轩$^1$}
\spAffiliation{$^1$复旦大学物理学系}
\date{}

\spAbstract{%
    建筑并非静态的物理存在, 而是承载社会关系、权力运作与文化记忆的复合体. 本文以圣依纳爵主教座堂为研究对象, 从“主体”与“场域”两个视角, 探讨宗教建筑如何成为符号资本运作的场所, 以及其在历史变迁中的多重意义. 作为“主体”, 教堂既是西方天主教会在华传播的象征性构筑, 也被中国地方文化塑造、重构, 形成了跨文化融合的动态形态; 作为“场域”, 教堂空间成为不同政治力量、社会群体争夺象征权威与文化话语权的竞技场. 从清末传教士的规划, 到民国时期的本土化策略, 再到当代的旅游化与文化遗产转向, 圣依纳爵主教座堂见证了空间符号的流变, 映射出全球化与本土性互动的复杂面向. 本文借助布尔迪厄的场域理论, 结合历史文献与空间分析, 尝试揭示宗教建筑如何在时间的流动中承载、形塑并折射文化资本的积累与再生产.
    \\\\
    \normalsize{\textbf{关键词}. \small{\bf{\song{符号资本, 文化记忆}}}}\leavevmode
}
\begin{document}
    \section{引言}
    建筑并非静态的物理存在, 而是承载社会关系、权力运作与文化记忆的复合体. 本文以圣依纳爵主教座堂为研究对象, 从“主体”与“场域”两个视角, 探讨宗教建筑如何成为符号资本运作的场所, 以及其在历史变迁中的多重意义. 作为“主体”, 教堂既是西方天主教会在华传播的象征性构筑, 也被中国地方文化塑造、重构, 形成了跨文化融合的动态形态; 作为“场域”, 教堂空间成为不同政治力量、社会群体争夺象征权威与文化话语权的竞技场. 从清末传教士的规划, 到民国时期的本土化策略, 再到当代的旅游化与文化遗产转向, 圣依纳爵主教座堂见证了空间符号的流变, 映射出全球化与本土性互动的复杂面向. 本文借助布尔迪厄的场域理论, 结合历史文献与空间分析, 尝试揭示宗教建筑如何在时间的流动中承载、形塑并折射文化资本的积累与再生产.

    \section{圣依纳爵主教座堂的历史建构}
    \lipsum[2]

    \section{符号资本的生产与转换}
    \lipsum[3]

    \section{圣依纳爵主教座堂的多重拆解}
    \lipsum[4]

    \section{符号资本的流动性}
    \lipsum[5]

    \section{结语}
    \lipsum[6]
\end{document}
