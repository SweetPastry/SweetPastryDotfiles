\section*{Problem 21}

    The definition of Laplace-Runge-Lenz vector is
    \begin{equation}
        \boldsymbol{A} = \boldsymbol{p} \times \boldsymbol{L} -mk\frac{\boldsymbol{r}}{r},
    \end{equation}
    for the purpose or considering how does the field influence this vector, differentiate it,
    \begin{align}
        \frac{\mathrm{d}\boldsymbol{A}}{\mathrm{d}t} &= \frac{\mathrm{d}\boldsymbol{p}}{\mathrm{d}t}\times\boldsymbol{L} + \boldsymbol{p}\times\frac{\mathrm{d}\boldsymbol{L}}{\mathrm{d}t} - mk\frac{\mathrm{d}}{\mathrm{d}t}\frac{\boldsymbol{r}}{r}\nonumber
        \\
        & = \left( \tikzmarknode{Kepler1}{\hlmath{gray}{\boldsymbol{F}_{\text{Kepler}}}} + \boldsymbol{F}_{\text{uniform}} \right)\times\boldsymbol{L} + \boldsymbol{p}\times\left(\boldsymbol{r}\times\boldsymbol{F}_{\text{uniform}}\right)\nonumber
        \\
        &\phantom{=}\tikzmarknode{Kepler2}{\hlmath{gray}{- mk\frac{\mathrm{d}}{\mathrm{d}t}\frac{\boldsymbol{r}}{r}}}\nonumber
        \\
        & = \boldsymbol{F}_{\text{uniform}}\times\boldsymbol{L} + \boldsymbol{p}\times\left(\boldsymbol{r}\times\boldsymbol{F}_{\text{uniform}}\right).
    \end{align}
    \begin{tikzpicture}[overlay,remember picture,>=stealth,nodes={align=left,inner ysep=1pt},<-]
    \node[anchor=north west,color=gray!67, yshift=+1em, xshift=4em] (jtext) at ($(Kepler1.south)!0.5!(Kepler2.south)$) {In the case of the Kepler potential,\\ these two terms cancel each other out.};
    \draw[<->,color=gray!57] (Kepler1.south east) -- (jtext.west) -- (Kepler2.east);
    \end{tikzpicture}