\begin{titlepage}
	\begin{tikzpicture}[font=\LARGE] 
		\def \tta{ -10.00000000000000 }
		\def \k{    -3.00000000000000 
		\def \l{     6.00000000000000 } % Defines the width  of the parallelepiped
		\def \d{     5.00000000000000 } % Defines the depth  of the parallelepiped
		\def \h{     7.00000000000000 } % Defines the heigth of the parallelepiped
		
		% The vertices A,B,C,D define the reference plan (vertical)
		\coordinate (A) at (0,0); 
		\coordinate (B) at ({-\h*sin(\tta)},{\h*cos(\tta)}); 
		\coordinate (C) at ({-\h*sin(\tta)-\d*sin(\k*\tta)},
							{\h*cos(\tta)+\d*cos(\k*\tta)}); 
		\coordinate (D) at ({-\d*sin(\k*\tta)},{\d*cos(\k*\tta)}); 
		
		% The vertices Ap,Bp,Cp,Dp define a plane translated from the 
		% reference plane by the width of the parallelepiped
		\coordinate (Ap) at (\l,0); 
		\coordinate (Bp) at ({\l-\h*sin(\tta)},{\h*cos(\tta)}); 
		\coordinate (Cp) at ({\l-\h*sin(\tta)-\d*sin(\k*\tta)},
							{\h*cos(\tta)+\d*cos(\k*\tta)}); 
		\coordinate (Dp) at ({\l-\d*sin(\k*\tta)},{\d*cos(\k*\tta)}); 
		
		% Marking the vertices of the tetrahedron (red)
		% and of the parallelepiped (black)
		\fill[black]  (A) circle [radius=2pt]; 
		\fill[red]    (B) circle [radius=2pt]; 
		\fill[black]  (C) circle [radius=2pt]; 
		\fill[red]    (D) circle [radius=2pt]; 
		\fill[red]   (Ap) circle [radius=2pt]; 
		\fill[black] (Bp) circle [radius=2pt]; 
		\fill[red]   (Cp) circle [radius=2pt]; 
		\fill[black] (Dp) circle [radius=2pt]; 
		
		% painting first the three visible faces of the tetrahedron
		\filldraw[draw=red,bottom color=red!50!black, top color=cyan!50]
		(B) -- (Cp) -- (D);
		\filldraw[draw=red,bottom color=red!50!black, top color=cyan!50]
		(B) -- (D)  -- (Ap);
		\filldraw[draw=red,bottom color=red!50!black, top color=cyan!50]
		(B) -- (Cp) -- (Ap);
		
		% Draw the edges of the tetrahedron
		\draw[red,-,very thick] (Ap) --  (D)
								(Ap) --  (B)
								(Ap) -- (Cp)
								(B)  --  (D)
								(Cp) --  (D)
								(B)  -- (Cp);
		
		% Draw the visible edges of the parallelepiped
		\draw [-,thin] (B)  --  (A)
					(Ap) -- (Bp)
					(B)  --  (C)
					(D)  --  (C)
					(A)  --  (D)
					(Ap) --  (A)
					(Cp) --  (C)
					(Bp) --  (B)
					(Bp) -- (Cp);
		
		% Draw the hidden edges of the parallelepiped
		\draw [gray,-,thin] (Dp) -- (Cp);
							(Dp) --  (D);
							(Ap) -- (Dp);
		
		% Name the vertices (the names are not consistent
		%  with the node name, but it makes the programming easier)
		\draw (Ap) node [right]           {$A$}
			(Bp) node [right, gray]     {$F$}
			(Cp) node [right]           {$D$}
			(C)  node [left,gray]       {$E$}
			(D)  node [left]            {$B$}
			(A)  node [left,gray]       {$G$}
			(B)  node [above left=+5pt] {$C$}
			(Dp) node [right,gray]      {$H$};
		
		% Drawing again vertex $C$, node (B) because it disappeared behind the edges.
		% Drawing again vertex $H$, node (Dp) because it disappeared behind the edges.
		\fill[red]   (B) circle [radius=2pt]; 
		\fill[gray] (Dp) circle [radius=2pt]; 
		
		% From the reference and this example one can easily draw 
		% the twin tetrahedron jointly to this one.
		% Drawing the edges of the twin tetrahedron
		% switching the p_s: A <-> Ap, etc...
		\draw[red,-,dashed, thin] (A)  -- (Dp)
								(A)  -- (Bp)
								(A)  --  (C)
								(Bp) -- (Dp)
								(C)  -- (Dp)
								(Bp) --  (C);
	\end{tikzpicture}
\end{titlepage}
