\begin{figure}[htb]
    \vspace{2\baselineskip}
    \begin{equation}
        \pr[\mathcal{R}(\tikzmarknode{ts}{\highlight{red}{$\tau_i$}},\tikzmarknode{js}{\highlight{blue}{$j$}})\in \mathcal{S}] \leq e^\epsilon \pr[\mathcal{R}(\tikzmarknode{td}{\highlight{red}{$\tau_i'$}},\tikzmarknode{jd}{\highlight{blue}{$j'$}})\in \mathcal{S}]
    \label{eq:dp_one_instance}
    \end{equation}
    \begin{tikzpicture}[overlay,remember picture,>=stealth,nodes={align=left,inner ysep=1pt},<-]
        % Ts to Td
        \node[anchor=south,color=red!67, yshift=2em] (ttext) at ($(ts.north)!0.5!(td.north)$) {\textbf{$\tau_i,\tau' \in \Gamma$}, \textbf{the set of Tasks}};
        \draw[<->,color=red!57] (ts.north) |- (ttext.south) -| (td.north);
        % js to jd
        \node[anchor=north,color=blue!67, yshift=-2em] (jtext) at ($(js.south)!0.5!(jd.south)$) {\textbf{$j,j'\in \mathbb{N}$}};
        \draw[<->,color=blue!57] (js.south) |- (jtext.north) -| (jd.south);
    \end{tikzpicture}
    \vspace{\baselineskip}
    \caption{A More Complex Example for Annotated Equations, this time inside a figure construct.}
\end{figure}    