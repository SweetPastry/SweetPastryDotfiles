\spChapter{甄士隐梦幻识通灵, 贾雨村风尘怀闺秀}
此开卷第一回也。作者自云:因曾历过一番梦幻之后,故将真事隐去,而借“通灵”之说,撰此《石头记》一书也。故曰“甄士隐”云云。但书中所记何事何人?自又云:“今风尘碌碌,一事无成,忽念及当日所有之女子,一一细考较去,觉其行止见识皆出于我之上。何我堂堂须眉,诚不若彼裙钗哉?实愧则有馀,悔又无益之大无可如何之日也!当此,则自欲将已往所赖天恩祖德,锦衣纨袴之时,饫甘餍肥之日,背父兄教育之恩,负师友规训之德,以至今日一技无成,半生潦倒之罪,编述一集,以告天下人:我之罪固不免,然闺阁中本自历历有人,万不可因我之不肖,自护己短,一并使其泯灭也。虽今日之茅椽蓬牖,瓦灶绳床,其晨夕风露,阶柳庭花,亦未有妨我之襟怀笔墨者。虽我未学,下笔无文,又何妨用假语村言,敷演出一段故事来,亦可使闺阁昭传,复可悦世之目,破人愁闷,不亦宜乎?”故曰“贾雨村云云。”

此回中凡用“梦”用“幻”等字,是提醒阅者眼目,亦是此书立意本旨。

列位看官:你道此书从何而来?说起根由虽近荒唐,细按则深有趣味。待在下将此来历注明,方使阅者了然不惑。

原来女娲氏炼石补天之时,于大荒山无稽崖炼成高经十二丈,方经二十四丈顽石三万六千五百零一块。娲皇氏只用了三万六千五百块,只单单剩了一块未用,便弃在此山青埂峰下。谁知此石自经煅炼之后,灵性已通,因见众石俱得补天,独自己无材不堪入选,遂自怨自叹,日夜悲号惭愧。

一日,正当嗟悼之际,俄见一僧一道远远而来,生得骨格不凡,丰神迥异,说说笑笑来至峰下,坐于石边高谈快论。先是说些云山雾海神仙玄幻之事,后便说到红尘中荣华富贵;此石听了,不觉打动凡心,也想要到人间去享一享这荣华富贵,但自恨粗蠢,不得已,便口吐人言,向那僧道说道:“大师,弟子蠢物,不能见礼了。适闻二位谈那人世间荣耀繁华,心切慕之。弟子质虽粗蠢,性却稍通;况见二师仙形道体,定非凡品,必有补天济世之材,利物济人之德。如蒙发一点慈心,携带弟子得入红尘,在那富贵场中,温柔乡里受享几年,自当永佩洪恩,万劫不忘也。”二仙师听毕,齐憨笑道:“善哉,善哉!那红尘中有却有些乐事,但不能永远依恃,况又有‘美中不足,好事多磨’八个字紧相连属,瞬息间则又乐极悲生,人非物换,究竟是到头一梦,万境归空,倒不如不去的好。”

这石凡心已炽,那里听得进这话去,乃复苦求再四。二仙知不可强制,乃叹道:“此亦静极思动,无中生有之数也。既如此,我们便携你去受享受享,只是到不得意时,切莫后悔。”石道:“自然,自然。”那僧又道:“若说你性灵,却又如此质蠢,并更无奇贵之处。如此也只好踮脚而已。也罢,我如今大施佛法助你助,待劫终之日,复还本质,以了此案。你道好否?”石头听了,感谢不尽。那僧便念咒书符,大展幻术,将一块大石登时变成一块鲜明莹洁的美玉,且又缩成扇坠大小的可佩可拿。那僧托于掌上,笑道:“形体倒也是个宝物了!还只没有实在的好处,须得再镌上数字,使人一见便知是奇物方妙。然后携你到那昌明隆盛之邦,诗礼簪缨之族,花柳繁华地,温柔富贵乡去安身乐业。”石头听了,喜不能禁,乃问:“不知赐了弟子哪几件奇处,又不知携了弟子到何地方?望乞明示,使弟子不惑。”那僧笑道:“你且莫问,日后自然明白的。”说着,便袖了这石,同那道人飘然而去,竟不知投奔何方何舍。

后来,又不知过了几世几劫,因有个空空道人访道求仙,忽从这大荒山无稽崖青埂峰下经过,忽见一大块石上字迹分明,编述历历。空空道人乃从头一看,原来就是无材补天,幻形入世,蒙茫茫大士渺渺真人携入红尘,历尽离合悲欢炎凉世态的一段故事。后面又有一首偈云:

无材可去补苍天,枉入红尘若许年。

此系身前身后事,倩谁记去作奇传?

诗后便是此石坠落之乡,投胎之处,亲自经历的一段陈迹故事。其中家庭闺阁琐事,以及闲情诗词倒还全备,或可适趣解闷,然朝代年纪、地舆邦国,却反失落无考。

空空道人遂向石头说道:“石兄,你这一段故事,据你自己说有些趣味,故编写在此,意欲问世传奇。据我看来,第一件,无朝代年纪可考;第二件,并无大贤大忠理朝廷治风俗的善政,其中只不过几个异样女子,或情或痴,或小才微善,亦无班姑蔡女之德能。我纵抄去,恐世人不爱看呢。”石头笑答道:“我师何太痴耶!若云无朝代可考,今我师竟假借汉唐等年纪添缀,又有何难?但我想,历来野史,皆蹈一辙,莫如我这不借此套者,反倒新奇别致,不过只取其事体情理罢了,又何必拘拘于朝代年纪哉!再者,市井俗人喜看理治之书者甚少,爱适趣闲文者特多。历来野史,或讪谤君相,或贬人妻女,奸淫凶恶,不可胜数。更有一种风月笔墨,其淫秽污臭,屠毒笔墨,坏人子弟,又不可胜数。至若佳人才子等书,则又千部共出一套,且其中终不能不涉于淫滥,以致满纸潘安、子建、西子、文君、不过作者要写出自己的那两首情诗艳赋来,故假拟出男女二人名姓,又必旁出一小人其间拨乱,亦如剧中之小丑然。且鬟婢开口即者也之乎,非文即理。故逐一看去,悉皆自相矛盾,大不近情理之话,竟不如我半世亲睹亲闻的这几个女子,虽不敢说强似前代书中所有之人,但事迹原委,亦可以消愁破闷;也有几首歪诗熟话,可以喷饭供酒。至若离合悲欢,兴衰际遇,则又追踪蹑迹,不敢稍加穿凿,徒为供人之目而反失其真传者。今之人,贫者日为衣食所累,富者又怀不足之心,纵然一时稍闲,又有贪淫恋色,好货寻愁之事,哪里去有工夫看那理治之书?所以我这一段故事,也不愿世人称奇道妙,也不定要世人喜悦检读,只愿他们当那醉淫饱卧之时,或避事去愁之际,把此一玩,岂不省了些寿命筋力?就比那谋虚逐妄,却也省了口舌是非之害,腿脚奔忙之苦。再者,亦令世人换新眼目,不比那些胡牵乱扯,忽离忽遇,满纸才人淑女、子建文君红娘小玉等通共熟套之旧稿。我师意为何如?”

空空道人听如此说,思忖半晌,将《石头记》再检阅一遍,因见上面虽有些指奸责佞贬恶诛邪之语,亦非伤时骂世之旨;及至君仁臣良父慈子孝,凡伦常所关之处,皆是称功颂德,眷眷无穷,实非别书之可比。虽其中大旨谈情,亦不过实录其事,又非假拟妄称,一味淫邀艳约、私订偷盟之可比。因毫不干涉时世,方从头至尾抄录回来,问世传奇。从此空空道人因空见色,由色生情,传情入色,自色悟空,遂易名为情僧,改《石头记》为《情僧录》。东鲁孔梅溪则题曰《风月宝鉴》。后因曹雪芹于悼红轩中披阅十载,增删五次,纂成目录,分出章回,则题曰《金陵十二钗》。并题一绝云:

满纸荒唐言,一把辛酸泪。

都云作者痴,谁解其中味?

出则既明,且看石上是何故事。按那石上书云:

当日地陷东南,这东南一隅有处曰姑苏,有城曰阊门者,最是红尘中一二等富贵风流之地。这阊门外有个十里街,街内有个仁清巷,巷内有个古庙,因地方窄狭,人皆呼作葫芦庙。庙旁住着一家乡宦,姓甄,名费,字士隐。嫡妻封氏,情性贤淑,深明礼义。家中虽不甚富贵,然本地便也推他为望族了。因这甄士隐禀性恬淡,不以功名为念,每日只以观花修竹、酌酒吟诗为乐,倒是神仙一流人品。只是一件不足:如今年已半百,膝下无儿,只有一女,乳名唤作英莲,年方三岁。

一日,炎夏永昼,士隐于书房闲坐,至手倦抛书,伏几少憩,不觉朦胧睡去。梦至一处,不辨是何地方。忽见那厢来了一僧一道,且行且谈。

只听道人问道:“你携了这蠢物,意欲何往?”那僧笑道:“你放心,如今现有一段风流公案正该了结,这一干风流冤家,尚未投胎入世。趁此机会,就将此蠢物夹带于中,使他去经历经历。”那道人道:“原来近日风流冤孽又将造劫历世去不成?但不知落于何方何处?”那僧笑道:“此事说来好笑,竟是千古未闻的罕事。只因西方灵河岸上三生石畔,有绛珠草一株,时有赤瑕宫神瑛侍者,日以甘露灌溉,这绛珠草始得久延岁月。后来既受天地精华,复得雨露滋养,遂得脱却草胎木质,得换人形,仅修成个女体,终日游于离恨天外,饥则食蜜青果为膳,渴则饮灌愁海水为汤。只因尚未酬报灌溉之德,故其五内便郁结着一段缠绵不尽之意。恰近日这神瑛侍者凡心偶炽,乘此昌明太平朝世,意欲下凡造历幻缘,已在警幻仙子案前挂了号。警幻亦曾问及,灌溉之情未偿,趁此倒可了结的。那绛珠仙子道:‘他是甘露之惠,我并无此水可还。他既下世为人,我也去下世为人,但把我一生所有的眼泪还他,也偿还得过他了。’因此一事,就勾出多少风流冤家来,陪他们去了结此案。”

那道人道:“果是罕闻。实未闻有还泪之说。想来这一段故事,比历来风月事故更加琐碎细腻了。”那僧道:“历来几个风流人物,不过传其大概以及诗词篇章而已;至家庭闺阁中一饮一食,总未述记。再者,大半风月故事,不过偷香窃玉,暗约私奔而已,并不曾将儿女之真情发泄一二。想这一干人入世,其情痴色鬼、贤愚不肖者,悉与前人传述不同矣。”那道人道:“趁此何不你我也去下世度脱几个,岂不是一场功德?”那僧道:“正合吾意,你且同我到警幻仙子宫中,将蠢物交割清楚,待这一干风流孽鬼下世已完,你我再去。如今虽已有一半落尘,然犹未全集。”道人道:“既如此,便随你去来。”

却说甄士隐俱听得明白,但不知所云“蠢物”系何东西。遂不禁上前施礼,笑问道:“二仙师请了。”那僧道也忙答礼相问。士隐因说道:“适闻仙师所谈因果,实人世罕闻者。但弟子愚浊,不能洞悉明白,若蒙大开痴顽,备细一闻,弟子则洗耳谛听,稍能警省,亦可免沉伦之苦。”二仙笑道:“此乃玄机不可预泄者。到那时不要忘我二人,便可跳出火坑矣。”士隐听了,不便再问。因笑道:“玄机不可预泄,但适云‘蠢物’,不知为何,或可一见否?”那僧道:“若问此物,倒有一面之缘。”说着,取出递与士隐。

士隐接了看时,原来是块鲜明美玉,上面字迹分明,镌着“通灵宝玉”四字,后面还有几行小字。正欲细看时,那僧便说已到幻境,便强从手中夺了去,与道人竟过一大石牌坊,上书四个大字,乃是“太虚幻境”。两边又有一幅对联,道是:

假作真时真亦假,无为有处有还无。

士隐意欲也跟了过去,方举步时,忽听一声霹雳,有若山崩地陷。士隐大叫一声,定睛一看,只见烈日炎炎,芭蕉冉冉,所梦之事便忘了大半。又见奶母正抱了英莲走来。士隐见女儿越发生得粉妆玉琢,乖觉可喜,便伸手接来,抱在怀内,斗他顽耍一回,又带至街前,看那过会的热闹。

方欲进来时,只见从那边来了一僧一道:那僧则癞头跣脚,那道则跛足蓬头,疯疯癫癫,挥霍谈笑而至。及至到了他门前,看见士隐抱着英莲,那僧便大哭起来,又向士隐道:“施主,你把这有命无运、累及爹娘之物,抱在怀内作甚?”士隐听了,知是疯话,也不去睬他。那僧还说:“舍我罢,舍我罢!”士隐不耐烦,便抱女儿撤身要进去,那僧乃指着他大笑,口内念了四句言词道:

惯养娇生笑你痴,菱花空对雪澌澌。

好防佳节元宵后,便是烟消火灭时。

士隐听得明白,心下犹豫,意欲问他们来历。只听道人说道:“你我不必同行,就此分手,各干营生去罢。三劫后,我在北邙山等你,会齐了同往太虚幻境销号。”那僧道:“最妙,最妙!”说毕,二人一去,再不见个踪影了。士隐心中此时自忖:这两个人必有来历,该试一问,如今悔却晚也。

这士隐正痴想,忽见隔壁葫芦庙内寄居的一个穷儒──姓贾名化、表字时飞、别号雨村者走了出来。这贾雨村原系胡州人氏,也是诗书仕宦之族,因他生于末世,父母祖宗根基已尽,人口衰丧,只剩得他一身一口,在家乡无益,因进京求取功名,再整基业。自前岁来此,又淹蹇住了,暂寄庙中安身,每日卖字作文为生,故士隐常与他交接。

当下雨村见了士隐,忙施礼陪笑道:“老先生倚门伫望,敢是街市上有甚新闻否?”士隐笑道:“非也。适因小女啼哭,引他出来作耍,正是无聊之甚,兄来得正妙,请入小斋一谈,彼此皆可消此永昼。”说着,便令人送女儿进去,自与雨村携手来至书房中。小童献茶。方谈得三五句话,忽家人飞报:“严老爷来拜。”士隐慌的忙起身谢罪道:“恕诳驾之罪,略坐,弟即来陪。”雨村忙起身亦让道:“老先生请便。晚生乃常造之客,稍候何妨。”说着,士隐已出前厅去了。

这里雨村且翻弄书籍解闷。忽听得窗外有女子嗽声,雨村遂起身往窗外一看,原来是一个丫鬟,在那里撷花,生得仪容不俗,眉目清明,虽无十分姿色,却亦有动人之处。雨村不觉看的呆了。

那甄家丫鬟撷了花,方欲走时,猛抬头见窗内有人,敝巾旧服,虽是贫窘,然生得腰圆背厚,面阔口方,更兼剑眉星眼,直鼻权腮。这丫鬟忙转身回避,心下乃想:“这人生的这样雄壮,却又这样褴褛,想他定是我家主人常说的什么贾雨村了,每有意帮助周济,只是没甚机会。我家并无这样贫窘亲友,想定是此人无疑了。怪道又说他必非久困之人。”如此想来,不免又回头两次。雨村见他回了头,便自为这女子心中有意于他,便狂喜不尽,自为此女子必是个巨眼英雄,风尘中之知己也。一时小童进来,雨村打听得前面留饭,不可久待,遂从夹道中自便出门去了。士隐待客既散,知雨村自便,也不去再邀。

一日,早又中秋佳节。士隐家宴已毕,乃又另具一席于书房,却自己步月至庙中来邀雨村。原来雨村自那日见了甄家之婢曾回顾他两次,自为是个知己,便时刻放在心上。今又正值中秋,不免对月有怀,因而口占五言一律云:

未卜三生愿,频添一段愁。

闷来时敛额,行去几回头。

自顾风前影,谁堪月下俦?

蟾光如有意,先上玉人楼。

雨村吟罢,因又思及平生抱负,苦未逢时,乃又搔首对天长叹,复高吟一联曰:

玉在椟中求善价,钗于奁内待时飞。

恰值士隐走来听见,笑道:“雨村兄真抱负不浅也!”雨村忙笑道:“不过偶吟前人之句,何敢狂诞至此。”因问:“老先生何兴至此?”士隐笑道:“今夜中秋,俗谓‘团圆之节’,想尊兄旅寄僧房,不无寂寥之感,故特具小酌,邀兄到敝斋一饮,不知可纳芹意否?”雨村听了,并不推辞,便笑道:“既蒙厚爱,何敢拂此盛情。”说着,便同士隐复过这边书院中来。

须臾茶毕,早已设下杯盘,那美酒佳肴自不必说。二人归坐,先是款斟漫饮,次渐谈至兴浓,不觉飞觥限斝起来。当时街坊上家家箫管,户户弦歌,当头一轮明月,飞彩凝辉,二人愈添豪兴,酒到杯干。雨村此时已有七八分酒意,狂兴不禁,乃对月寓怀,口号一绝云:

时逢三五便团圆,满把晴光护玉栏。

天上一轮才捧出,人间万姓仰头看。

士隐听了,大叫:“妙哉!吾每谓兄必非久居人下者,今所吟之句,飞腾之兆已见,不日可接履于云霓之上矣。可贺,可贺!”乃亲斟一斗为贺。雨村因干过,叹道:“非晚生酒后狂言,若论时尚之学,晚生也或可去充数沽名,只是目今行囊路费一概无措,神京路远,非赖卖字撰文即能到者。”士隐不待说完,便道:“兄何不早言。愚每有此心,但每遇兄时,兄并未谈及,愚故未敢唐突。今既及此,愚虽不才,‘义利’二字却还识得。且喜明岁正当大比,兄宜作速入都,春闱一战,方不负兄之所学也。其盘费馀事,弟自代为处置,亦不枉兄之谬识矣!”当下即命小童进去,速封五十两白银,并两套冬衣。又云:“十九日乃黄道之期,兄可即买舟西上,待雄飞高举,明冬再晤,岂非大快之事耶!”雨村收了银衣,不过略谢一语,并不介意,仍是吃酒谈笑。那天已交了三更,二人方散。

士隐送雨村去后,回房一觉,直至红日三竿方醒。因思昨夜之事,意欲再写两封荐书与雨村带至神都,使雨村投谒个仕宦之家为寄足之地。因使人过去请时,那家人去了回来说:“和尚说,贾爷今日五鼓已进京去了,也曾留下话与和尚转达老爷,说‘读书人不在黄道黑道,总以事理为要,不及面辞了。’”士隐听了,也只得罢了。

真是闲处光阴易过,倏忽又是元宵佳节矣。士隐命家人霍启抱了英莲去看社火花灯,半夜中,霍启因要小解,便将英莲放在一家门槛上坐着。待他小解完了来抱时,哪有英莲的踪影?急得霍启直寻了半夜,至天明不见,那霍启也就不敢回来见主人,便逃往他乡去了。那士隐夫妇,见女儿一夜不归,便知有些不妥,再使几人去寻找,回来皆云连音响皆无。夫妻二人,半世只生此女,一旦失落,岂不思想,因此昼夜啼哭,几乎不曾寻死。看看的一月,士隐先就得了一病,当时封氏孺人也因思女构疾,日日请医疗治。

不想这日三月十五,葫芦庙中炸供,那些和尚不加小心,致使油锅火逸,便烧着窗纸。此方人家多用竹篱木壁者,大抵也因劫数,于是接二连三,牵五挂四,将一条街烧得如火焰山一般。彼时虽有军民来救,那火已成了势,如何救得下?直烧了一夜,方渐渐的熄去,也不知烧了几家。只可怜甄家在隔壁,早已烧成一片瓦砾场了。只有他夫妇并几个家人的性命不曾伤了。急得士隐惟跌足长叹而已。只得与妻子商议,且到田庄上去安身。偏值近年水旱不收,鼠盗蜂起,无非抢田夺地,鼠窃狗偷,民不安生,因此官兵剿捕,难以安身。士隐只得将田庄都折变了,便携了妻子与两个丫鬟投他岳丈家去。

他岳丈名唤封肃,本贯大如州人氏,虽是务农,家中都还殷实。今见女婿这等狼狈而来,心中便有些不乐。幸而士隐还有折变田地的银子未曾用完,拿出来托他随分就价薄置些须房地,为后日衣食之计。那封肃便半哄半赚,些须与他些薄田朽屋。士隐乃读书之人,不惯生理稼穑等事,勉强支持了一二年,越觉穷了下去。封肃每见面时,便说些现成话,且人前人后又怨他们不善过活,只一味好吃懒作等语。士隐知投人不着,心中未免悔恨,再兼上年惊唬,急忿怨痛,已有积伤,暮年之人,贫病交攻,竟渐渐的露出那下世的光景来。

可巧这日拄了拐杖挣挫到街前散散心时,忽见那边来了一个跛足道人,疯癫落脱,麻屣鹑衣,口内念着几句言词,道是:

世人都晓神仙好,惟有功名忘不了!

古今将相在何方?荒冢一堆草没了。

世人都晓神仙好,只有金银忘不了!

终朝只恨聚无多,及到多时眼闭了。

世人都晓神仙好,只有娇妻忘不了!

君生日日说恩情,君死又随人去了。

世人都晓神仙好,只有儿孙忘不了!

痴心父母古来多,孝顺儿孙谁见了?

士隐听了,便迎上来道:“你满口说些什么?只听见些‘好’‘了’‘好’‘了’。”那道人笑道:“你若果听见‘好’‘了’二字,还算你明白。可知世上万般,好便是了,了便是好。若不了,便不好;若要好,须是了。我这歌儿,便名《好了歌》”士隐本是有宿慧的,一闻此言,心中早已彻悟。因笑道:“且住!待我将你这《好了歌》解注出来何如?”道人笑道:“你解,你解。”士隐乃说道:

陋室空堂,当年笏满床;衰草枯杨,曾为歌舞场。

蛛丝儿结满雕梁,绿纱今又糊在蓬窗上。说什么脂正浓,粉正香,如何两鬓又成霜?

昨日黄土陇头送白骨,今宵红灯帐底卧鸳鸯。

金满箱,银满箱,转眼乞丐人皆谤。

正叹他人命不长,那知自己归来丧!

训有方,保不定日后作强梁。择膏粱,谁承望流落在烟花巷!

因嫌纱帽小,致使锁枷扛,昨怜破袄寒,今嫌紫蟒长。

乱烘烘你方唱罢我登场,反认他乡是故乡。甚荒唐,到头来都是为他人作嫁衣裳!

那疯跛道人听了,拍掌笑道:“解得切,解得切!”士隐便说一声“走罢!”将道人肩上褡裢抢了过来背着,竟不回家,同了疯道人飘飘而去。当下烘动街坊,众人当作一件新闻传说。封氏闻得此信,哭个死去活来,只得与父亲商议,遣人各处访寻,哪讨音信?无奈何,少不得依靠着他父母度日。幸而身边还有两个旧日的丫鬟伏侍,主仆三人,日夜作些针线发卖,帮着父亲用度。那封肃虽然日日抱怨,也无可奈何了。

这日,那甄家大丫鬟在门前买线,忽听街上喝道之声,众人都说新太爷到任。丫鬟于是隐在门内看时,只见军牢快手,一对一对的过去,俄而大轿抬着一个乌帽猩袍的官府过去。丫鬟倒发了个怔,自思这官好面善,倒像在哪里见过的。于是进入房中,也就丢过不在心上。至晚间,正待歇息之时,忽听一片声打的门响,许多人乱嚷,说:“本府太爷差人来传人问话。”封肃听了,唬得目瞪口呆,不知有何祸事。