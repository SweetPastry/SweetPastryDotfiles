\chapter{甄士隐梦幻识通灵, 贾雨村风尘怀闺秀}

列位看官:你道此书从何而来?说起根由虽近荒唐,(甲戌侧批:自站地步。自首荒唐,妙!) 细按则深有趣味。待在下将此来历注明,方使阅者了然不惑。

原来女娲氏炼石补天之时,(甲戌侧批:补天济世,勿认真,用常言。) 于大荒山(甲戌侧批:荒唐也。) 无稽崖(甲戌侧批:无稽也。) 练成高经十二丈、(甲戌侧批:总应十二钗。) 方经二十四丈(甲戌侧批:照应副十二钗。) 顽石三万六千五百零一块。娲皇氏只用了三万六千五百块,(甲戌侧批:合周天之数。蒙侧批:数足,偏遗我。“不堪入选”句中透出心眼。) 只单单的剩了一块未用,(甲戌侧批:剩了这一块便生出这许多故事。使当日虽不以此补天,就该去补地之坑陷,使地平坦,而不有此一部鬼话。) 便弃在此山青埂峰下。(甲戌眉批:妙!自谓落堕情根,故无补天之用。) 谁知此石自经煅炼之后,灵性已通,(甲戌侧批:煅炼后性方通,甚哉!人生不能学也。) 因见众石俱得补天,独自己无材不堪入选,遂自怨自叹,日夜悲号惭愧。

一日,正当嗟悼之际,俄见一僧一道远远而来,生得骨格不凡,丰神迥别,(蒙双行夹批:这是真像,非幻像也。该批:作者自己形容。) 说说笑笑来至峰下,坐于石边高谈快论。先是说些云山雾海神仙玄幻之事,后便说到红尘中荣华富贵。

此石听了,不觉打动凡心,也想要到人间去享一享这荣华富贵,但自恨粗蠢,不得已,便口吐人言,(甲戌侧批:竟有人问口生于何处,其无心肝,可笑可恨之极。) 向那僧道说道:“大师,弟子蠢物,(甲戌侧批:岂敢岂敢。) 不能见礼了。适闻二位谈那人世间荣耀繁华,心切慕之。弟子质虽粗蠢,(甲戌侧批:岂敢岂敢。) 性却稍通,况见二师仙形道体,定非凡品,必有补天济世之材,利物济人之德。如蒙发一点慈心,携带弟子得入红尘,在那富贵场中、温柔乡里受享几年,自当永佩洪恩,万劫不忘也。”

二仙师听毕,齐憨笑道:“善哉,善哉!那红尘中有却有些乐事,但不能永远依恃,况又有‘美中不足,好事多魔’八个字紧相连属,瞬息间则又乐极悲生,人非物换,究竟是到头一梦,万境归空。(甲戌侧批:四句乃一部之总纲。) 倒不如不去的好。” 这石凡心已炽,哪里听得进这话去,乃复苦求再四。

二仙知不可强制,乃叹道:“此亦静极思动,无中生有之数也。既如此,我们便携你去受享受享,只是到不得意时,切莫后悔。” 石道:“自然,自然。” 那僧又道:“若说你性灵,却又如此质蠢,并更无奇贵之处,如此也只好踮脚而已。(甲戌侧批:煅炼过尚与人踮脚,不学者又当如何?) 也罢,我如今大施佛法助你助,待劫终之日,复还本质,以了此案。(甲戌侧批:妙!佛法亦须偿还,况世人之债乎?近之赖债者来看此句。所谓游戏笔墨也。) 你道好否?” 石头听了,感谢不尽。

那僧便念咒书符,大展幻术,将一块大石登时变成一块鲜明莹洁的美玉,且又缩成扇坠大小的可佩可拿。(甲戌侧批:奇诡险怪之文,有如髯苏《石钟》《赤壁》用幻处。) 那僧托于掌上,笑道:“形体倒也是个宝物了!(甲戌侧批:自愧之语。蒙双行夹批:世上人原自据看得见处为凭。) 还只没有实在的好处,(甲戌侧批:好极!今之金玉其外败絮其中者,见此大不欢喜。) 须得再镌上数字,使人一见便知是奇物方妙。(甲戌侧批:世上原宜假,不宜真也。) 然后携你到那昌明隆盛之邦,诗礼簪缨之族,花柳繁华地,温柔富贵乡去安身乐业。” 

石头听了,喜不能禁,乃问:“不知赐了弟子那几件奇处,又不知携了弟子到何地方?望乞明示,使弟子不惑。”