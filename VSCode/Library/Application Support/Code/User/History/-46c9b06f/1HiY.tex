\documentclass[aspectratio=43]{beamer}
%%
\usepackage{tikz}
\usetikzlibrary{backgrounds}
\usetikzlibrary{arrows,shapes}
\usetikzlibrary{tikzmark}
\usetikzlibrary{calc}
\usepackage{tcolorbox}
\newcommand{\hlmath}[2]{\colorbox{#1!70}{$\displaystyle #2$}}
\newcommand{\hltext}[2]{\colorbox{#1!70}{#2}}
%%
\usepackage[english]{babel}
\usepackage{extarrows}
% \usepackage{ctex}
\usepackage{xeCJK}
\input{preamble}
\title{\bfseries
    Rayleigh's Dissipation Function at Work
}
\subtitle{
    Classic Mechanics Term Report
}
\author[H.-X. Lin]{\bfseries Hai-Xuan Lin (林海轩)}
\institute[\textit{
    Department of Physics, Fudan University
}]
{\textit{
    Department of Physics, Fudan University
}}
\date{
    \today
}
\begin{document}

    \frame{\titlepage}

    \begin{frame}{Summary}
        \tableofcontents
    \end{frame}

    \section{Introduction}
    \frame{\sectionpage}
        \begin{frame}{This Report is Based on Thesis: \cite{Minguzzi2015}}
            \begin{figure}
                \centering
                \includegraphics[width=0.95\textwidth]{ThesisInfo.png}
            \end{figure}
        \end{frame}

        \begin{frame}{Raw Lagrange Equation}
            \uncover<+->{As we all know, in case of potential $V$ is independent with velocity of $\dot{q}$ we have Lagrange Equation}
            \uncover<+->{
            \only<1,2>{\begin{equation}
                \frac{\mathrm{d}}{\mathrm{d}t}\frac{\partial T}{\partial \dot{q}^k}-\frac{\partial T}{\partial q^k}+\frac{\partial V}{\partial q^k}=Q_{k}^{\left( \mathrm{nc} \right)}.
            \end{equation}}}
            \only<3>{\begin{equation}
                \frac{\mathrm{d}}{\mathrm{d}t}\frac{\partial T}{\partial \dot{q}^{\tikzmarknode{k}{\hlmath{red}{k}}}}-\frac{\partial T}{\partial q^k}+\frac{\partial V}{\partial q^k}=Q_{k}^{\left( \mathrm{nc} \right)}.
            \end{equation}\begin{tikzpicture}[overlay, remember picture, >=stealth, nodes={align=left, inner ysep=2pt}, <-]
                \path (k.south) ++ (0,-0.8em) node[anchor=north west, color=red!67] (k node){Superscript, not exponent};
                \draw [color=red!57](k.south) |- ([xshift=+0.3ex, color=red]k node.south east);
            \end{tikzpicture}}
            \only<4>{\begin{equation}
                \frac{\mathrm{d}}{\mathrm{d}t}\frac{\partial T}{\partial \dot{q}^k}-\frac{\partial T}{\partial q^k}+\frac{\partial V}{\partial q^k}=Q_{k}^{\left(\tikzmarknode{nc}{\hlmath{red}{ \mathrm{nc}}} \right)}.
            \end{equation}\begin{tikzpicture}[overlay, remember picture, >=stealth, nodes={align=left, inner ysep=2pt}, <-]
                \path (nc.south) ++ (0,-1.8em) node[anchor=north east, color=red!67] (nc node){means non-conservative};
                \draw [color=red!57](nc.south) |- ([xshift=-0.3ex, color=red]nc node.south west);
            \end{tikzpicture}}
        \end{frame}

        \begin{frame}{Rayleigh's Disppative Protential}
            \uncover<+->{If the non-conservative force linear correlate with velocities, consider particle $a$,}

            \uncover<+->{
            \only<1, 2>{\begin{equation}
                F_{ai}=-K_{i j}v_{a}^{\tikzmarknode{j}{j}},
            \end{equation}}}

            \only<3>{\begin{equation}
                F_{ai}=-K_{i \tikzmarknode{j1}{\hlmath{red}{j}}}v_{a}^{\tikzmarknode{j2}{\hlmath{red}{j}}},
            \end{equation}\begin{tikzpicture}[overlay,remember picture,>=stealth,nodes={align=left,inner ysep=1pt},<-]
            \node[anchor=north,color=red!67, yshift=-2em] (jtext) at ($(j1.south)!0.5!(j2.south)$) {Einstein Summation Convention};
            \draw[<->,color=red!57] (j1.south) |- (jtext.north) -| (j2.south);
            \end{tikzpicture}}

            \only<4, 5, 6, 7, 8>{\begin{equation}
                F_{ai}=-K_{i j}v_{a}^{\tikzmarknode{j}{j}},
            \end{equation}

            \uncover<5, 6, 7, 8>{and define Rayleigh disspative potential}\uncover<6, 7, 8>{
            \begin{equation}
                R=\frac{1}{2}\sum_a{K_{ij}v_{a}^{i}v_{a}^{j}}.
            \end{equation}}}

            \uncover<7, 8>{We can verify that}

            \uncover<8>{
                \begin{equation}
                    Q_{k}^{\left( \mathrm{nc} \right)}=-\sum_a{\nabla _{\mathbf{v}_a}R\frac{\partial \mathbf{v}_a}{\partial \dot{q}^k}+Q_{k}^{\left( \mathrm{nc}^{\prime} \right)}}=-\frac{\partial R}{\partial \dot{q}^k}+Q_{k}^{\left( \mathrm{nc}^{\prime} \right)}.
                \end{equation}
            }
        \end{frame}

        \begin{frame}{Lagrange-Rayleigh Equation}
            \only<1>{\begin{equation}
                \frac{\mathrm{d}}{\mathrm{d}t} \frac{\partial T}{\partial \dot{q}^k}-\frac{\partial T}{\partial q^k}+\frac{\partial V}{\partial q^k}=-\frac{\partial R}{\partial \dot{q}^k}+Q_{k}^{\left( \mathrm{nc'} \right)}.\label{LReq}
            \end{equation}}

            \only<2>{\begin{equation}
                \frac{\mathrm{d}}{\mathrm{d}t} \frac{\partial T}{\partial \dot{q}^k}-\frac{\partial T}{\partial q^k}+\frac{\partial V}{\partial q^k}=-\frac{\partial \tikzmarknode{R}{\hlmath{red}{R}}}{\partial \dot{q}^k}+Q_{k}^{\left( \mathrm{nc'} \right)}.
            \end{equation}\begin{tikzpicture}[overlay, remember picture, >=stealth, nodes={align=left, inner ysep=2pt}, <-]
                \path (R.south) ++ (0,-2.0em) node[anchor=north east, color=red!67] (R node){Rayleigh Dissipation Protential};
                \draw [color=red!57](R.south) |- ([xshift=-0.3ex, color=red]R node.south west);
            \end{tikzpicture}}

            \only<3>{\begin{equation}
                \frac{\mathrm{d}}{\mathrm{d}t} \frac{\partial T}{\partial \dot{q}^k}-\frac{\partial T}{\partial q^k}+\frac{\partial V}{\partial q^k}=-\frac{\partial R}{\partial \dot{q}^k}+Q_{k}^{\left( \tikzmarknode{nc}{\hlmath{red}{\mathrm{nc'}}}  \right)}.
            \end{equation}\begin{tikzpicture}[overlay, remember picture, >=stealth, nodes={align=left, inner ysep=2pt}, <-]
                \path (nc.south) ++ (0,-2.0em) node[anchor=north east, color=red!67] (nc node){non-conservative force except Rayleigh Linear Force};
                \draw [color=red!57](nc.south) |- ([xshift=-0.3ex, color=red]nc node.south west);
            \end{tikzpicture}}
        \end{frame}

        \begin{frame}{What did this paper do?}
            \uncover<+->{The Rayleigh Protential is only suitable for linear friction force.}

            \vspace{\baselineskip}

            \uncover<+->{This paper promote Rayleigh Protential to general case when consider contact friction.}
        \end{frame}

    \section{Generalized Friction Protential}
    \frame{\sectionpage}

    \subsection{Derivation Process}
    \frame{\subsectionpage}
    \begin{frame}{Generalized Rayleigh Protential}
        \uncover<+->{Surfaces contact friction can be write as}

        \uncover<+->{\begin{equation}
            \mathbf{F}=-N\mu \left( v \right) \hat{\mathbf{v}},
        \end{equation}}     

        \vspace{-0.8\baselineskip}

        \uncover<+->{\only<1, 2, 3>{and let $\mathbf{v}_{a}^{\left( \mathrm{r} \right)}=\mathbf{v}_a-\mathbf{v}_b$. Here we allow the 2 surfaces move respectively, with consistent the same normal.}
        \only<4>{and let $\mathbf{v}_{a}^{\left( \mathrm{r} \right)}=\mathbf{v}_{\tikzmarknode{a}{\hlmath{red}{a}}}-\mathbf{v}_b$. Here we allow the 2 surfaces move respectively, with consistent the same normal.\begin{tikzpicture}[overlay, remember picture, >=stealth, nodes={align=left, inner ysep=2pt}, <-]
        \path (a.south) ++ (0,-2.0em) node[anchor=north west, color=red!67] (a node){Denote $S_1$ is the surface we concern and \\ $a$ is a point above it};
        \draw [color=red!57](a.south) |- ([xshift=+0.3ex, color=red]a node.south east);
        \end{tikzpicture}}
        \only<5>{and let $\mathbf{v}_{a}^{\left( \mathrm{r} \right)}=\mathbf{v}_a-\mathbf{v}_{\tikzmarknode{b}{\hlmath{red}{b}}}$. Here we allow the 2 surfaces move respectively, with consistent the same normal.\begin{tikzpicture}[overlay, remember picture, >=stealth, nodes={align=left, inner ysep=2pt}, <-]
            \path (b.south) ++ (0,-2.0em) node[anchor=north west, color=red!67] (b node){Denote $S_2$ is reference surface \\ and $b$ is the contact point with $a$};
            \draw [color=red!57](b.south) |- ([xshift=+0.3ex, color=red]b node.south east);
        \end{tikzpicture}}
        }
    \end{frame}

    \begin{frame}{Generalized Rayleigh Protential}
        \uncover<+->{
            So the tiny friction of point $a$ is
        }\uncover<+->{\begin{equation}
            \mathbf{F}_{a}^{\left( \mathrm{nc} \right)}=-p_a\Delta A\mu \left( v_{a}^{\left( \mathrm{r} \right)} \right) {\hat{\mathbf{v}}_a}^{\left( \mathrm{r} \right)}.
        \end{equation}}

        \uncover<+->{\begin{theorem}
            \uncover<+->{If we define $R$ as}
            \uncover<+->{\begin{equation}
                R=\Delta A\sum_a{p_a \int^{v_{a}^{\left( \mathrm{r} \right)}} {\hspace{-0.8em}\mu \mathrm{d}v}},
            \end{equation}}
            \uncover<+->{so that indeed}
            \uncover<+->{\begin{equation}
                -\nabla _{\mathbf{v}_a}R=-\frac{\partial R}{\partial v_{a}^{\left( \mathrm{r} \right)}}\nabla _{\mathbf{v}_a}v_{a}^{\left( \mathrm{r} \right)}=-\Delta Ap_a\mu \left( v_{a}^{\left( \mathrm{r} \right)} \right) \hat{\mathbf{v}}_{a}^{\left( \mathrm{r} \right)}=\mathbf{F}_{a}^{\left( \mathrm{nc} \right)}.
            \end{equation}}
        \end{theorem}}
    \end{frame}

    \begin{frame}{Generalized Rayleigh Protential}
        \addtocounter{framenumber}{-1}

        So the tiny friction of point $a$ is
        \begin{equation}
        \mathbf{F}_{a}^{\left( \mathrm{nc} \right)}=-p_a\Delta A\mu \left( v_{a}^{\left( \mathrm{r} \right)} \right) {\hat{\mathbf{v}}_a}^{\left( \mathrm{r} \right)}.
        \end{equation}

        \begin{theorem}
            If we define $R$ as
            \begin{equation}
              \tikzmarknode{R}{\hlmath{red}{R=\Delta A\sum_a{p_a \int^{v_{a}^{\left( \mathrm{r} \right)}} {\hspace{-0.8em}\mu \mathrm{d}v}}}}  ,
            \end{equation}
            \begin{tikzpicture}[overlay, remember picture, >=stealth, nodes={align=left, inner ysep=2pt}, <-]
                \path (R.south) ++ (0,-0.8em) node[anchor=north west, color=red!67] (R node){Generalized Rayleigh Protential};
                \draw [color=red!57](R.south) |- ([xshift=+0.3ex, color=red]R node.south east);
            \end{tikzpicture}
            so that indeed
            \begin{equation}
                -\nabla _{\mathbf{v}_a}R=-\frac{\partial R}{\partial v_{a}^{\left( \mathrm{r} \right)}}\nabla _{\mathbf{v}_a}v_{a}^{\left( \mathrm{r} \right)}=-\Delta Ap_a\mu \left( v_{a}^{\left( \mathrm{r} \right)} \right) \hat{\mathbf{v}}_{a}^{\left( \mathrm{r} \right)}=\mathbf{F}_{a}^{\left( \mathrm{nc} \right)}.
            \end{equation}
        \end{theorem}
    \end{frame}

    \begin{frame}{Generalized Rayleigh Protential}
        \uncover<+->{\begin{corollary}
            Let $a$ to be infinitely tiny as well as $\Delta A\to \mathrm{d}A$,
        \uncover<+->{\begin{equation}
            R=\int_{S_1}{\mathrm{d}Ap\left( x \right) \int_{\,\,}^{v^{\left( \mathrm{r} \right)}\left( x \right)}{\hspace{-1.2em}\mu \mathrm{d}v}}.\label{CalR}
        \end{equation}}
        \end{corollary}}
    \end{frame}

    \subsection{Application}
    \frame{\subsectionpage}

    \begin{frame}{A Rotating Disk}
        \uncover<+->{\begin{example}
            \uncover<+->{A disk of mass $m$ and radius $r$ placed on horizontal surface of friction coefficient $\mu$.} \uncover<+->{At time $t=0$ disk has angular velocity $\omega$ about its center and zero translational velocity. Find out at which time $\tau$ it will stop.}
        \end{example}}
        \uncover<+->{\begin{solution}
            \uncover<+->{The Generalized Rayleigh Protential can be written according to formula \eqref{CalR},}\uncover<+->{\begin{equation}
                R=\int_{S_1}{\mathrm{d}A\frac{mg}{\pi r^2}}\int_{\,\,}^{\rho \dot{\varphi}}{\mu \mathrm{d}v}=\frac{\mu mg}{\pi r^2}\int_{S_1}{\mathrm{d}A}\int_{\,\,}^{\rho \dot{\varphi}}{\mu \mathrm{d}v}=\frac{2}{3}\mu mgR\dot{\varphi}.
            \end{equation}}
        \end{solution}}
    \end{frame}

    \begin{frame}
        \uncover<+->{\begin{block}{}
        \uncover<+->{And the Lagrangian $L=T=\frac{1}{4}mr^2\dot{\varphi}^2$. Take them into Lagrange-Rayleigh equation \eqref{LReq},} \uncover<+->{\begin{equation}
            \frac{\mathrm{d}}{\mathrm{d}t}\left( \frac{1}{2}mr^2\dot{\varphi} \right) =-\frac{2}{3}\mu mgR.
        \end{equation}}
        \uncover<+->{
            Thus $\ddot{\varphi}=-\frac{4}{3}\frac{\mu g}{r}$, and
        }\uncover<+->{\begin{equation*}
            \tau =\frac{3\omega R}{4\mu g}.
        \end{equation*}}
    \end{block}}
    \end{frame}

    \frame{\centering\Large How about more complex case?}

    \begin{frame}{A Rotating Disk}
        \uncover<+->{\begin{lemma}{\uncover<+->{(Reye's Hypothesis)}}
            \uncover<+->{If the system exhibits point-by-point identical wear on contact surface, the $\mu$ satisfied}\uncover<+->{
            \begin{equation}
                \mathrm{d}N=\frac{k}{v^{\left( \mathrm{r} \right)}\left( x \right)}\mathrm{d}A,
            \end{equation}}\uncover<+->{the $k$ is a coefficient satisfing the normalization condition.}
        \end{lemma}}
    \end{frame}

    \begin{frame}{A Rotating Disk}
        \uncover<+->{
            \begin{example}
                \uncover<+->{
                    Find out $\tau$ under Reye's Hypothesis.
                }
            \end{example}
        }

        \uncover<+->{
            \begin{solution}
                \uncover<+->{
                    Using formula \eqref{CalR} again, \uncover<+->{
                        \begin{align}
                            R&=\int_{S_1}{\mathrm{d}Ap\left( x \right) \int_{\,\,}^{\rho \dot{\varphi}}{\mu \mathrm{d}v}}\nonumber
                            \\
                            &=\mu \int_{S_1}{\mathrm{d}Ap\left( x \right)}\int_{\,\,}^{\rho \dot{\varphi}}{\mathrm{d}v}\nonumber
                        \end{align}
                    }
                }
            \end{solution}
        }
    \end{frame}

    \begin{frame}
    	\uncover<+->{
            \begin{block}{}
                \begin{align}
                    &\uncover<+->{=\mu \int_{S_1}{\mathrm{d}N}\rho \dot{\varphi}\nonumber}
                    \\
                    &\uncover<+->{=\mu k\int_{S_1}{\frac{\mathrm{d}A}{\rho \dot{\varphi}}}\rho \dot{\varphi},\qquad \left( \int_D{\mathrm{d}N}=k\int_D{\frac{1}{v^{\left( \mathrm{r} \right)}}\mathrm{d}A}=mg \right) \nonumber}
                    \\
                    &\uncover<0>{=\mu mg\left[ \int_D{\frac{1}{\rho \dot{\varphi}}\rho \mathrm{d}\rho \mathrm{d}\varphi} \right] ^{-1}\int_D{\mathrm{d}A}=\frac{1}{2}\mu mgr\dot{\varphi}\nonumber}
                \end{align}
            \end{block}}
    \end{frame}
    
    \addtocounter{framenumber}{-1}
    \begin{frame}
    	\uncover<+->{
    		\begin{block}{}
    			\begin{align}
    				&{=\mu \int_{S_1}{\mathrm{d}N}\rho \dot{\varphi}\nonumber}
    				\\
    				&{\tikzmarknode{Re}{\hlmath{red}{=}}\mu k\int_{S_1}{\frac{\mathrm{d}A}{\rho \dot{\varphi}}}\rho \dot{\varphi},\qquad \left( \int_D{\mathrm{d}N}=k\int_D{\frac{1}{v^{\left( \mathrm{r} \right)}}\mathrm{d}A}=mg \right) \nonumber}
    				\\
    				&\uncover<+->{=\mu mg\left[ \int_D{\frac{1}{\rho \dot{\varphi}}\rho \mathrm{d}\rho \mathrm{d}\varphi} \right] ^{-1}\int_D{\mathrm{d}A}=\frac{1}{2}\mu mgr\dot{\varphi}}
    			\end{align}
    	\end{block}}
        \begin{tikzpicture}[overlay, remember picture, >=stealth, nodes={align=left, inner ysep=2pt}, <-]
            \path (Re.south) ++ (0,-7.0em) node[anchor=north west, color=red!67] (Re node){Reye's Hypothesis};
            \draw [color=red!57](Re.south) |- ([xshift=+0.3ex, color=red]Re node.south east);
        \end{tikzpicture}
    \end{frame}
    
	\addtocounter{framenumber}{-1}
    \begin{frame}
            \begin{block}{}
                \begin{align}
                    &{=\mu \int_{S_1}{\mathrm{d}N}\rho \dot{\varphi}\nonumber}
                    \\
                    &{=\mu k\int_{S_1}{\frac{\mathrm{d}A}{\rho \dot{\varphi}}}\rho \dot{\varphi},\qquad \left( \int_D{\mathrm{d}N}=k\int_D{\frac{1}{v^{\left( \mathrm{r} \right)}}\mathrm{d}A}=mg \right) \nonumber}
                    \\
                    &{=\mu mg\left[ \int_D{\frac{1}{\rho \dot{\varphi}}\rho \mathrm{d}\rho \mathrm{d}\varphi} \right] ^{-1}\int_D{\mathrm{d}A}=\frac{1}{2}\mu mgr\dot{\varphi}\nonumber}
                \end{align}
            \end{block}
            So $\tau=\frac{\omega r}{\mu g}$, using Lagrange-Rayleigh equation.
    \end{frame}
    


    \section*{References}
        \begin{frame}[allowframebreaks]{References}
            \nocite{*}
            \printbibliography
       \end{frame}

    \section{}
    \begin{frame}{}
        \centering
            \Huge\bfseries
        \textcolor{orange}{The End}
    \end{frame}
\end{document}