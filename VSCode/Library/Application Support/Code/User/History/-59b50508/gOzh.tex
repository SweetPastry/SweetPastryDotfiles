\spChapter[正式写作示例\\Formal writing demo]{正式写作示例, Formal writing demo}
    \section{Math Writing, 数理类写作}
        \lipsum[1-2]
        \begin{definition}
            If \(a\) and \(b\) are even integers, then their sum \(a+b\) is even.
        \end{definition}
        \lipsum[3-4]
        \begin{lemma}[Some lemma]
            Since \(a\) and \(b\) are even, there exist integers \(k\) and \(m\) such that 
            \[
            a = 2k \quad \text{and} \quad b = 2m.
            \]
            Then,
            \[
            a+b = 2k + 2m = 2(k+m).
            \]
            Because \(k+m\) is an integer, \(a+b\) is divisible by 2 and therefore even.
        \end{lemma}
        \lipsum[5]
        \begin{corollary}[Corollary][Newton Corollary]
            If \(a\) and \(b\) are odd integers, then their product \(ab\) is odd.
        \end{corollary}
        \lipsum[6-7]
        \begin{corollary}[theorem][Newton@ImNewton!YouAreNewton]
            \lipsum[8-9]
        \end{corollary}
        \lipsum[10-11]
        \begin{proposition}[中文命题][ZhongWenMingTi@中文命题显示!中文命题子显示]
            壬戌之秋,七月既望,苏子与客泛舟游于赤壁之下。清风徐来,水波不兴。举酒属客,诵明月之诗,歌窈窕之章。少焉,月出于东山之上,徘徊于斗牛之间。白露横江,水光接天。纵一苇之所如,凌万顷之茫然。浩浩乎如冯虚御风,而不知其所止;飘飘乎如遗世独立,羽化而登仙。j j
        \end{proposition}