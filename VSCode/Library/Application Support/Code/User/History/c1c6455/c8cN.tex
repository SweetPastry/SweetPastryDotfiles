\documentclass[
    ref = refDemo,
    style = 2
]{spBeamer}
\spTitle{\texttt{spBeamer} Demo}
\spAuthor{Sweet Pastry}
\spAuthorInShort{SP}
\spAffiliation{Fudan University, Shanghai, China}
\spAffiliationInShort{FDU}

\begin{document}
    \section{Introduction}
        \begin{frame}{Logo}
            % \begin{figure}
                % \centering
                \includegraphics[width=\textwidth]{cover.pdf}
                % \caption{Cover, Sweet Pastry}
            % \end{figure}
        \end{frame}

    \section{Math}
        \begin{frame}{\texttt{uncover} command}
            \uncover<+->{%
            \begin{definition}[Linear Functional]
                A linear functional on a vector space \( X \) over \( \mathbb{R} \) or \( \mathbb{C} \) is a map \( f : X \to \mathbb{R} \) (or \( \mathbb{C} \)) such that:
                \[
                f(\alpha x + \beta y) = \alpha f(x) + \beta f(y) \quad \forall x,y \in X, \ \alpha, \beta \in \mathbb{R} \ (\text{or } \mathbb{C})
                \]
            \end{definition}
            \uncover<+->{
            \begin{definition}[Sublinear Functional]
                A map \( p: X \to \mathbb{R} \) is called sublinear if:
                \[
                p(x + y) \leq p(x) + p(y) \quad \text{and} \quad p(\lambda x) = \lambda p(x) \quad \forall x,y \in X, \ \lambda \geq 0
                \]
            \end{definition}
            }}  
        \end{frame}

        \begin{frame}
            \begin{theorem}[Hahn-Banach, Normed Space Version]
                Let \( X \) be a normed space, \( M \) a subspace of \( X \), and \( f: M \to \mathbb{R} \) a bounded linear functional. Suppose \( p: X \to \mathbb{R} \) is a sublinear functional such that:
                \[
                f(x) \leq p(x) \quad \forall x \in M
                \]
                Then, there exists an extension \( F: X \to \mathbb{R} \) of \( f \) such that:
                \[
                F(x) \leq p(x) \quad \forall x \in X
                \]
                and \( F \) is linear and bounded.
            \end{theorem}
        \end{frame}
        
        \begin{frame}
            \uncover<+->{%
            The proof uses Zorn's Lemma and proceeds by extending \( f \) step by step to larger subspaces. The key idea is to define:
            \uncover<+->{%
            \[
            F(x + \alpha y) = F(x) + \alpha f(y)
            \]
            \uncover<+->{%
            for some \( y \notin M \) and ensure the extension satisfies the sublinear constraint:
            \uncover<+->{%
            \[
            F(x + y) \leq p(x + y)
            \]
            \uncover<+->{%
            Applying Zorn's Lemma to the collection of all extensions leads to the existence of the desired functional \( F \).
            }}}}}
        \end{frame}
\end{document}