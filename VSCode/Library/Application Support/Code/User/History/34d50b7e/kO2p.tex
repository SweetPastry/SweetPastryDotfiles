\documentclass[
    author = Sweet Pastry,
    affiliation = Fudan University,
    column = onecolumn,
    ref = refDemo.bib,
    lang = cn
]{spArticle}

\spAbstract{
    神圣既远,祸乱相寻,学士大夫有以生民为虑、王道为心者绝少,宋没益不可问。先生禀绝世之资,慨焉以斯文自任。会文明启运,千载一时。深维上天所以生我之意,与古圣贤之所讲求,直欲排洪荒而开二帝,去杂霸而见三王,又推其馀以淑来禩,伊周孔孟合为一人,将旦暮遇之。此非学而有以见性分之大全不能也。既而时命不偶,遂以九死成就一个是,完天下万世之责。其扶持世教,信乎不愧千秋正学者也。考先生在当时已称程、朱复出,後之人反以一死抹过先生一生若心,谓节义与理学是两事,出此者入彼,至不得与扬雄、吴草庐论次并称。於是成仁取义之训为世大禁,而乱臣贼子将接踵於天下矣,悲夫!或言先生之忠至矣,而十族与殉,无乃伤於激乎?余曰:“先生只自办一死,其激而及十族,十族各办其一死耳。普天之下莫非王土,十族众乎?而不当死乎?惟先生平日学问,断断乎臣尽忠,子尽孝,一本於良心之所固有者。率天下而趋之,至数十年之久,几於风移世变,一日乃得透此一段精光,不可掩遏。盖至诚形著,动变之理宜然,而非人力之所几及也,虽谓先生为中庸之道可也。”
}

\begin{document}
    \section{曹月川端}
    先生之学,不由师传,特从古册中翻出古人公案,深有悟於造化之理,而以月川体其传,反而求之吾心,即心是极,即心之动静是阴阳,即心之日用酬酢是五行变合,而一以事心为入道之路。故其见虽彻而不玄,学愈精而不杂,虽谓先生为今之濂溪可也。乃先生自谱,其於斯道,至四十而犹不胜其渺茫浩瀚之苦,又十年恍然一悟,始知天下无性外之物,而性无不在焉,所谓太极之理即此而是。盖见道之难如此,学者慎毋轻言悟也哉!

    按先生门人彭大司马泽尝称:我朝一代文明之盛、经济之学,莫盛于刘诚意、宋学士,至道统之传,则断自渑池曹先生始。上章请从祀孔子庙庭。事在正德中。愚谓方正学而後,斯道之绝而复续者,实赖有先生一人。薛文清亦闻先生之风而起者。

    \section{薛敬轩瑄}
        \subsection{薛敬轩瑄}
            愚按前辈论一代理学之儒,惟先生无间言,非以实践之儒欤?然先生为御史,在宣、正两朝,未尝铮铮一论事;景皇易储,先生时为大理,亦无言。或云先生方转饷贵州,及于萧愍之狱,系当朝第一案,功罪是非,而先生仅请从未减,坐视忠良之死而不之救,则将焉用彼相焉。就事相提,前日之不谏是,则今日之谏非,两者必居一於此。而先生亦已愧不自得,乞身去矣。然先生於道,於古人全体大用尽多缺陷,特其始终进退之节有足称者,则亦成其为“文清”而已。阅先生《读书录》,多兢兢检点言行间,所谓“学贵践履”,意盖如此。或曰:“‘七十六年无一事,此心惟觉性天通。’先生晚年闻道,未可量也。”
        \subsection{Analysis}
            \lipsum[9-11]

    \section{Conclusion}
        \lipsum[12-14]
\end{document}