\spChapter{Preface}
    As a \LaTeX\ beginner, I often struggled to understand the meanings behind its complex commands and to clearly distinguish between concepts such as \emph{macros} and \emph{commands}. The typographical programming nature of \LaTeX\ significantly differs from traditional logic-based programming languages, which added to the learning curve.

    My introduction to \LaTeX\ occurred during my first year at university. At that time, I was completely unfamiliar with human-computer interaction, as my pre-university education strictly prohibited the use of electronic devices—such usage was considered either cheating or a violation of school rules. As a physics student, I frequently encountered extensive mathematical calculations involving abstract Greek letters and various symbolic notations. When it came to submitting homework or laboratory reports, handwritten drafts often posed several challenges: they were difficult for teaching assistants to read and grade, and they created an impression of being either disrespectful or lacking proficiency in modern computational tools.

    Driven by necessity, I began learning \LaTeX. After over a year of practice and exploration, I decided to consolidate my knowledge and experiences into a collection of templates designed to simplify the creation of \LaTeX\ documents. My goal was to minimize the complexity of the main document source file and to provide a reusable, easy-to-understand solution for users facing similar challenges. These templates are now publicly available on \href{https://github.com/SweetPastry/spTemplate?tab=CC-BY-4.0-1-ov-file}{GitHub}, along with detailed documentation to assist potential users. The target audience includes those who, like myself, are new to \LaTeX\ but require its capabilities for academic or professional purposes.

    It is my sincere hope that these templates and accompanying instructions can help others overcome the steep initial learning curve of \LaTeX\ and enable them to produce polished, professional documents efficiently.
    
    This open-source repository is licensed under \textbf{CC BY 4.0}. Users are free to use, modify, distribute, and commercially exploit the content, provided \href{https://github.com/SweetPastry/spTemplate/blob/main/LICENSE}{proper attribution} is given.

    \vspace{2cm}
    \begin{flushright}
        \textit{Sweet Pastry} \\[1em]
        \textit{Fudan University}
    \end{flushright}