\documentclass[tikz, varwidth,border=1cm]{article}
\usepackage[paperwidth=594mm, paperheight=841mm, margin=0mm, noheadfoot, nomarginpar]{geometry}
\usepackage{graphicx}
\usepackage{amsmath}
\usepackage{amssymb}
\usepackage{calligra}
\usepackage{textcomp}
\usepackage{tikz}
\usetikzlibrary{backgrounds}
\usetikzlibrary{arrows,shapes}
\usetikzlibrary{tikzmark}
\usetikzlibrary{calc}
\usetikzlibrary{patterns}
\usepackage{tcolorbox}
\usepackage{smartdiagram}
\usepackage[dvipsnames]{xcolor}
\usepackage{xeCJK}

\newcommand{\hlmath}[2]{\colorbox{#1!17}{$\displaystyle #2$}}
\newcommand{\hltext}[2]{\colorbox{#1!17}{#2}}
\newcommand{\song}{\CJKfamily{song}}

\title{\textbf{\scalebox{5}{强化学习在求解数学问题中的应用探索}}}\date{}
\begin{document}
    \maketitle
    \par

    \raisebox{-20cm}{
    \begin{minipage}[t]{0.45\textwidth}
        \scalebox{3}{%
            \smartdiagram[bubble diagram]{\textbf{强化学习}\\ \small{Reinforcement Learning}, \textbf{符号回归}\\ \small{Symbolic Regression}, \textbf{样本利用率}\\\small{Sample utilization}, \textbf{稀疏奖励}\\\small{Sparse reward}, \textbf{泛化性}\\\small{Generalization}, \textbf{数学问题}\\\small{Mathematics}}
            }
    \end{minipage}
    \hfill
    \begin{minipage}[t]{0.45\textwidth}
        \raisebox{15cm}{%
            \scalebox{4}{%
                % \fbox{\emph{\shortstack{\textbf{成员}:\ 林海轩\ 杨奕乐\ 刘子逸\ 翟成蹊\\ \textbf{指导老师}:\ 杜恺}}}
                \fbox{\parbox{7cm}{%
                    \emph{\textbf{成员}}:\ \tikzmarknode{Me}{\hltext{blue}{林海轩}}\ \tikzmarknode{Mem}{\hltext{red}{杨奕乐\ 刘子逸\ 翟成蹊}}
                    \par
                    \begin{tikzpicture}[overlay, remember picture, >=stealth, nodes={align=left, inner ysep=2pt}, <-]
                        \path (Me.north) ++ (0,+1.2em) node[anchor=south east, color=blue!67] (Me node){课题组学生负责人\\物理学系};
                        \draw [color=blue!57](Me.north) |- ([xshift=-0.3ex, color=blue]Me node.south west);
                    \end{tikzpicture}
                    \begin{tikzpicture}[overlay, remember picture, >=stealth, nodes={align=left, inner ysep=2pt}, <-]
                        \path (Mem.north) ++ (0,+1.2em) node[anchor=south west, color=red!67] (Mem node){课题组学生成员\\数学科学学院};
                        \draw [color=red!57](Mem.north) |- ([xshift=-0.3ex, color=red]Mem node.south east);
                    \end{tikzpicture}

                    \par

                    \emph{\textbf{指导老师}}:\ \tikzmarknode{Te}{\hltext{OliveGreen}{魏柯}}
                    \par
                    \begin{tikzpicture}[overlay, remember picture, >=stealth, nodes={align=left, inner ysep=2pt}, <-]
                        \path (Te.west) ++ (6.5em,0em) node[anchor=west, color=OliveGreen!100] (Te node){课题组指导老师\\大数据学院教授};
                        \draw [color=OliveGreen!100](Te.east) -- ([xshift=+0.3ex, color=OliveGreen]Te node.west);
                    \end{tikzpicture}

                    \par

                    \emph{\textbf{课题简介}}:\ \small{\song 本课题聚焦强化学习在数学问题求解(如组合优化)中的应用,探讨其样本利用率、稀疏奖励、泛化性等关键问题,并尝试改进现有方法. 通过研究,培养学生的科研创新与跨学科思维能力,为人工智能相关工作奠定基础. }
                }}
            }
        }
    \end{minipage}
    }

    \par

    \raisebox{0cm}{
        \begin{minipage}[t]{0.45\textwidth}
            \raisebox{20cm}{%
                \scalebox{4}{%
                    \fbox{\parbox{6cm}{
                        \emph{\textbf{\large 成果介绍}}:\\\\ {\small{%
                            在本课题中, 我们利用最小二乘法定义损失函数, 通过求解其极值来解决简单符号回归问题.
                            \par
                            设给定数据集 $D=\left\{\left(x_i, y_i\right)\right\}_{i=1}^{n}$, 其中 $x_i\in\mathbb{R}^d$ 是输入向量, $y_i\in\mathbb{R}$ 是输出. 对候选集 $f\in\mathbb{F}$, 定义损失函数
                            $$
                            l\left(f\right) = \sum_{i=1}^{n}\left(f\left(x_i\right)-y_i\right)^2,
                            $$
                            于是目标函数为 $f^* = \mathrm{arg}\min\limits
                            _{f\in\mathbb{F}}l\left(f\right)$.
                            \par
                            我们采用了一种线性方法, 设目标函数为候选集全体成员的线性表出 $f\left(x, \theta\right)=\sum\limits_{h_j\in\mathbb{F}}\theta_jh_j\left(x\right)$, 将损失函数求梯度得到方程组,
                            $$
                            \scalebox{0.8}{$\left( \begin{array}{c}
                                y_1\\
                                \vdots\\
                                y_n\\
                            \end{array} \right) =\left( \begin{matrix}
                                1&		f_1\left( x_1 \right)&		\cdots&		f_k\left( x_1 \right)\\
                                \vdots&		\vdots&		\ddots&		\vdots\\
                                1&		f_1\left( x_n \right)&		\cdots&		f_k\left( x_n \right)\\
                            \end{matrix} \right) \left( \begin{array}{c}
                                \theta _0\\
                                \vdots\\
                                \theta _k\\
                            \end{array} \right),$}
                            $$
                            借用矩阵语言, 我们找到了系数的表达式 $\boldsymbol{\theta}=\left(\boldsymbol{U}^\mathrm{T}\boldsymbol{U}\right)^{-1}\boldsymbol{U}^\mathrm{T}\boldsymbol{Y}$.
                            \par
                            我们进行了如下实验检验我们的成果: 设拟合目标为 $f\left(x\right)=\sin x^2\cdot\cos x-1$, 库 1 为纯多项式库 $\left\{1, x, x^2, \dots\right\}$, 库 2 额外添加了一些非线性函数 $\sin x, \cos x, \mathrm{e}^x, \mathrm{sigmoid}\left(x\right)$.
                        }}
                    }}
                }
            }
        \end{minipage}
        \hfill
        \begin{minipage}[t]{0.45\textwidth}
            \raisebox{20cm}{\scalebox{2}{%
                % Definition of blocks:
                \tikzset{%
                block/.style    = {draw, thick, rectangle, minimum height = 3em,
                minimum width = 3em},
                sum/.style      = {draw, circle, node distance = 2cm}, % Adder
                input/.style    = {coordinate}, % Input
                output/.style   = {coordinate} % Output
                }
                % Defining string as labels of certain blocks.
                \newcommand{\suma}{\Large$+$}
                \newcommand{\inte}{$\displaystyle \int$}
                \newcommand{\derv}{\huge$\frac{d}{dt}$}

                \begin{tikzpicture}[auto, thick, node distance=2cm, >=triangle 45]
                    \draw
                        % Drawing the blocks of first filter :
                        node at (0,0)[right=-3mm]{\Large \textopenbullet}
                        node [input, name=input1] {}
                        node [sum, right of=input1] (suma1) {\suma}
                        node [block, right of=suma1] (inte1) {\inte}
                            node at (6.8,0)[block] (Q1) {\Large $Q_1$}
                            node [block, below of=inte1] (ret1) {\Large$T_1$};
                        % Joining blocks.
                        % Commands \draw with options like [->] must be written individually
                        \draw[->](input1) -- node {$X(Z)$}(suma1);
                        \draw[->](suma1) -- node {} (inte1);
                        \draw[->](inte1) -- node {} (Q1);
                        \draw[->](ret1) -| node[near end]{} (suma1);
                        % Adder
                    \draw
                        node at (5.4,-4) [sum, name=suma2] {\suma}
                            % Second stage of filter
                        node at  (1,-6) [sum, name=suma3] {\suma}
                        node [block, right of=suma3] (inte2) {\inte}
                        node [sum, right of=inte2] (suma4) {\suma}
                        node [block, right of=suma4] (inte3) {\inte}
                        node [block, right of=inte3] (Q2) {\Large$Q_2$}
                        node at (9,-8) [block, name=ret2] {\Large$T_2$}
                    ;
                        % Joining the blocks of second filter
                        \draw[->] (suma3) -- node {} (inte2);
                        \draw[->] (inte2) -- node {} (suma4);
                        \draw[->] (suma4) -- node {} (inte3);
                        \draw[->] (inte3) -- node {} (Q2);
                        \draw[->] (ret2) -| (suma3);
                        \draw[->] (ret2) -| (suma4);
                            % Third stage of filter:
                        % Defining nodes:
                    \draw
                        node at (11.5, 0) [sum, name=suma5]{\suma}
                        node [output, right of=suma5]{}
                        node [block, below of=suma5] (deriv1){\derv}
                        node [output, right of=suma5] (sal2){}
                    ;
                        % Joining the blocks:
                        \draw[->] (suma2) -| node {}(suma3);
                        \draw[->] (Q1) -- (8,0) |- node {}(ret1);
                        \draw[->] (8,0) |- (suma2);
                        \draw[->] (5.4,0) -- (suma2);
                        \draw[->] (Q1) -- node {}(suma5);
                        \draw[->] (deriv1) -- node {}(suma5);
                        \draw[->] (Q2) -| node {}(deriv1);
                            \draw[<->] (ret2) -| node {}(deriv1);
                            \draw[->] (suma5) -- node {$Y(Z)$}(sal2);
                            % Drawing nodes with \textbullet
                    \draw
                        node at (8,0) {\textbullet}
                        node at (8,-2){\textbullet}
                        node at (5.4,0){\textbullet}
                            node at (5,-8){\textbullet}
                            node at (11.5,-6){\textbullet}
                            ;
                        % Boxing and labelling noise shapers
                        \draw [color=gray,thick](-0.5,-3) rectangle (9,1);
                        \node at (-0.5,1) [above=5mm, right=0mm] {\textsc{first-order noise shaper}};
                        \draw [color=gray,thick](-0.5,-9) rectangle (12.5,-5);
                        \node at (-0.5,-9) [below=5mm, right=0mm] {\textsc{second-order noise shaper}};
                \end{tikzpicture}
            }}
            \scalebox{4}{\fbox{\parbox{6cm}{%
                我们发现, 使用线性方法,即使函数库包括了原本的目标函数,也会被函数
                库中的其他函数干扰而无法得到精确结果. 以上实验表明,符号回归问题无法使用线性方法完全解决,因此,本小组认为,在符号回归问题中引入强化学习方法是值得考虑的.
            }}}
        \end{minipage}
    }
\end{document}